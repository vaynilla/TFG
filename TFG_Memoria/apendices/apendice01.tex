% !TeX root = ../libro.tex
% !TeX encoding = utf8
\chapter{Manual de instalación y acceso}\label{ap:apendice1}
\section{Requisitos previos}
Antes de instalar y ejecutar el programa, debemos tener instalado lo siguiente en el sistema:
\begin{itemize}
	\item Python
	\item PIP (gestor de paquetes de Python)
	\item Flask, NumPy, SciPy, Flask-CORS (librerías de Python)
	\item npm y Node.js
	\item React
\end{itemize}
\section{Instrucciones de instalación}
\subsection{Primera parte: API}
\begin{enumerate}
	\item Clonamos el repositorio del TFG desde GitHub y accedemos al de la API:
	\begin{verbatim}
		git clone https://github.com/vaynilla/TFG.git
		cd API_TFG
	\end{verbatim}
	\item Instalamos las dependencias de Python utilizando PIP:
	\begin{verbatim}
		pip install -r requirements.txt
	\end{verbatim}
	\item Ejecutamos el servidor de la API:
	\begin{verbatim}
		python app.py
	\end{verbatim}
	La API ahora debería estar en funcionamiento y accesible en la dirección URL especificada (por ejemplo, http://localhost:$5000$).
\end{enumerate}
\subsection{Segunda parte: Simulador}
\begin{enumerate}
	\item Accedemos al repositorio del simulador:
	\begin{verbatim}
		cd simulador
	\end{verbatim}
	\item Instalamos las dependencias del proyecto utilizando npm:
	\begin{verbatim}
		npm install
	\end{verbatim}
	\item Iniciamos la aplicación:
	\begin{verbatim}
		npm start
	\end{verbatim}
\end{enumerate}
\subsection{Acceso a la aplicación}
Una vez hemos completado los pasos anteriores, podremos acceder a la aplicación desde el navegador web:
\begin{itemize}
	\item Backend (API): La API estará disponible en la URL especificada por Flask (por ejemplo, http://localhost:$5000$).
	\item Frontend (React): La aplicación frontend estará disponible en http://localhost:$3000$ por defecto, a menos que se especifique lo contrario.
\end{itemize}

\endinput
%------------------------------------------------------------------------------------
% FIN DEL APÉNDICE. 
%------------------------------------------------------------------------------------
