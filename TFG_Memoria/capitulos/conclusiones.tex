% !TeX root = ../libro.tex
% !TeX encoding = utf8
\chapter{Conclusiones y trabajo futuro}
\section{Conclusiones}
EL ORDEN DE LOS PÁRRAFOS HAY QUE COMPROBARLO
\begin{itemize}
	\item 1-2 párrafos presentando el contexto (¿en qué ambito de aplicación está ese trabajo?) se enfoca en el desarrollo de un simulador innovador que emplea tecnologías de vanguardia para predecir las temperaturas internas de edificios en función de diversos factores externos. Este proyecto combina la potencia de React en el frontend con Python y Flask en el backend, ofreciendo una solución integral para la simulación precisa del comportamiento térmico de estructuras edificadas. La aplicación se fundamenta en un riguroso estudio de sistemas de ecuaciones diferenciales e integrales, los cuales han sido meticulosamente analizados y aplicados para modelar de manera efectiva las complejas interacciones térmicas dentro de un entorno arquitectónico. Las conclusiones derivadas de este trabajo representan un avance significativo en el campo de la simulación térmica de edificaciones, ofreciendo nuevas perspectivas para la optimización del diseño y la eficiencia energética en la construcción de estructuras habitables.
	\item 1-2 párrafos de motivación (¿qué pasa?, ¿hay algo por solucionar?) La idea de crear una herramienta útil para estudiar el comportamiento de las distintas temperaturas que transcurren a lo largo de un día en un edificio, aplicando un modelo matemático y dando la posibilidad de que otras personas puedan utilizarlo en su beneficio.
	\item 1 párrafo donde se plantee el objetivo general del TFG ("En este trabajo se pretende...") En este trabajo se pretende diseñar, implementar y desplegar un sistema que, haciendo uso de los modelos matemáticos, permita al usuario simular la temperatura en el interior de un edificio de acuerdo a la configuración de ciertos parámetros. realizar un estudio de algunos de los métodos de resolución o aproximación de la solución de ciertas ecuaciones integrales de Fredholm o Volterra, así como la aplicación de algunos de ellos al estudio de un modelo matemático para la distribución de la temperatura en el interior de un edificio con varias estancias. Además, diversas simulaciones permitirán analizar de manera realista la eficiencia energética de un edificio en términos de los parámetros y funciones que describen el modelo.
	\item 1-2 párrafos para mencionar las contribuciones en la rama de matemáticas (¿qué has hecho?) La parte matemática nos ha servido para sentar una base sobre las ecuaciones integrales de Fredholm y Volterra, ser capaces de entender y aplicar varios métodos para su resolución tanto escalar como vectorialmente, y poder aplicarlo en un contexto concreto como es el de simular las temperaturas internas de un edificio. OTRO PÁRRAFO: Hemos estudiado en profundidad las ecuaciones lineales integrales de Volterra de segunda clase, CON SU existencia y unicidad, etc... Además de estudiar la relación entre las ecuaciones diferenciales y ecuaciones de Volterra, TAMBIÉN hemos visto varios modelos de edificios y formas de resolverlos que nos servirían en la implementación del simulador para resolver los sistemas de forma óptima.
	\item 1-2 párrafos para mencionar las contribuciones en la rama de informática (¿qué propones?, ¿qué has implementado y para qué?) En cuanto a la parte informática, la idea es crear una aplicación software que sirva como simulador y que permita representar una gran variedad de edificios de forma intuitiva para que cualquier usuario pueda hacer uso de esta herramienta, así como mostrar de forma gráfica y directa la simulación de las temperaturas en el edificio a lo largo del día. OTRO PÁRRAFO: QUÉ HE IMPLEMENTADO PARA ELLO: Para ello hemos implementado una API como servicio web que nos permite resolver sistemas de ecuaciones diferenciales, usando para ello tecnologías como Flask y Python, junto con un entorno web desarrollado con React con el que poder enviar y recibir peticiones a la API, mostrando todos los datos de forma gráfica, permitiéndonos incluso descargar los resultados para otro posible uso.
	\item 1 párrafo de resumen de que con las contribuciones anteriores (matemáticas + informática) has conseguido cumplir el objetivo general, mencionando que a través de la consecución de una serie de objetivos específicos. -> Gracias a las contribuciones que hemos realizado tanto por la parte matemática como informática, hemos conseguido cumplir con el objetivo general del trabajo, crear un simulador funcional que estudie el comportamiento de las temperaturas internas de un edificio gracias a un modelo matemático concreto. ¿INCLUIR LOS OBJETIVOS ESPECÍFICOS QUE HEMOS IDO CUMPLIENDO? Hemos seguido unos objetivos específicos para llegar hasta ese punto, 
\end{itemize}
\section{Trabajo futuro}
ME FALTA TRABAJO FUTURO DE MATEMATICAS (PREGUNTAR A MANOLO)
El simulador es plenamente funcional e ilustra la implementación y uso de los modelos matemáticos propuestos a través de un conjunto de servicios. No obstante, se plantean posibles mejoras que sin duda serían de gran utilidad para el usuario final:
\begin{itemize}
	\item Cambiar la forma en la que se introducen los parámetros del edificio y hacer un panel en el que de forma gráfica y dinámica se vaya creando el edificio, y podamos ir añadiendo o quitando elementos tales como habitaciones, personas, etc... Esto sería una mejora en el diseño de la página además de permitir un uso más sencillo de la misma.
	\item Podríamos acceder a una API de tiempo meteorológico para no tener que introducir la temperatura externa a mano y obtenerla automáticamente. Además de trabajar con las próximas $24$ horas, podríamos estudiar cualquier día o intervalo de tiempo en específico.
	\item Actualmente utilizamos una arquitectura monolítica para el desarrollo, la cual no permite mucha escalabilidad, así que podríamos migrarlo todo a microservicios y a la nube para que sea más escalable.
\end{itemize}
\endinput
%------------------------------------------------------------------------------------
% FIN DEL CAPÍTULO. 
%------------------------------------------------------------------------------------