% !TeX root = ../libro.tex
% !TeX encoding = utf8
\chapter{Conclusiones y trabajo futuro}
\section{Conclusiones}
El trabajo se enfoca en el desarrollo de un simulador que emplea tecnologías actuales para predecir las temperaturas internas de edificios en función de diversos factores externos. Este proyecto es dinámico gracias al uso de React en el frontend, y con el uso de Python y Flask para el desarrollo del backend, ofreciendo una solución para la simulación precisa del comportamiento térmico de estructuras edificadas. La aplicación está fundamentada en un riguroso estudio de sistemas de ecuaciones integrales junto con varios tipos de métodos para resolverlos, los cuales han sido analizados y aplicados para modelar de manera efectiva las temperaturas dentro de un edificio. Las conclusiones de este trabajo representan un avance en el campo de la simulación térmica de edificaciones, ofreciendo nuevas perspectivas para la optimización del diseño y la eficiencia energética en la construcción de estructuras habitables.\\

La motivación principal para el trabajo es crear una herramienta útil y gráfica para estudiar el comportamiento de las distintas temperaturas que transcurren a lo largo de un día en un edificio, siendo capaces de modificar los parámetros y la estructura del edificio de forma cómoda y dinámica e ir comprobando cómo varían las temperaturas,  aplicando para ello un modelo matemático y además, dar la posibilidad de que otras personas puedan utilizarlo en su beneficio.\\

En este trabajo se pretende diseñar, implementar y desplegar un sistema que, haciendo uso de los modelos matemáticos, permita al usuario simular la temperatura en el interior de un edificio de acuerdo a la configuración de ciertos parámetros. Además, se pretende realizar un estudio general de algunos tipos de ecuaciones integrales, así como métodos de resolución o aproximación de la solución para algunas de ellas, y la aplicación de un modelo matemático para la distribución de la temperatura en el interior de un edificio con varias estancias. Además, gracias a las simulaciones se podría analizar de manera realista la eficiencia energética de un edificio en términos de los parámetros y funciones que describen el modelo, pero este análisis se escapa del objeto del trabajo.\\

La parte matemática nos ha servido para sentar una base sobre las ecuaciones integrales, ser capaces de entender y aplicar varios métodos para su resolución escalar, también vectorial en el caso de Volterra, y poder aplicarlo en un contexto concreto como es el de simular las temperaturas internas de un edificio. En particular, nos hemos centrado en las ecuaciones lineales integrales de Volterra de segunda clase, estudiando y analizando la existencia y unicidad de sus soluciones. Hemos visto la relación entre las ecuaciones diferenciales y ecuaciones lineales integrales de Volterra, por último, hemos ejemplificado varios modelos de edificios y distintas formas de resolver sus sistemas asociados que nos servirán como guía en el desarrollo del simulador para resolver los sistemas de forma óptima.\\

En cuanto a la parte informática, la idea es crear una aplicación software que sirva como simulador y que permita representar una gran variedad de edificios de forma intuitiva para que cualquier usuario pueda hacer uso de esta herramienta, así como mostrar de forma gráfica y directa la simulación de las temperaturas en el edificio a lo largo del día.

Para ello hemos implementado una API como servicio web que nos permite resolver sistemas de ecuaciones diferenciales, usando tecnologías como Flask y Python, junto con un entorno web desarrollado con React con el que poder enviar y recibir peticiones a la API, además de mostrar todos los datos y soluciones de forma gráfica, permitiéndonos incluso descargar los resultados para poder utilizarlos con otro fin.\\

Gracias a las contribuciones que hemos realizado tanto por la parte matemática como informática, hemos conseguido cumplir con el objetivo general del trabajo, crear un simulador funcional que estudie el comportamiento de las temperaturas internas de un edificio basándose en un modelo matemático concreto.
\section{Trabajo futuro}
El simulador es plenamente funcional e ilustra la implementación y uso de los modelos matemáticos propuestos a través de un conjunto de servicios. No obstante, se plantean posibles mejoras que sin duda serían de gran utilidad para el usuario final:
\begin{itemize}
	\item Cambiar la forma en la que se introducen los parámetros del edificio y hacer un panel en el que de forma gráfica y dinámica se vaya creando el edificio, y podamos ir añadiendo o quitando elementos tales como habitaciones, personas, etc... Esto sería una mejora en el diseño de la página además de permitir un uso más sencillo de la misma.
	\item Podríamos acceder a una API de tiempo meteorológico para no tener que introducir la temperatura externa a mano y obtenerla automáticamente. Además de trabajar con las próximas $24$ horas, podríamos estudiar cualquier día o intervalo de tiempo en específico.
	\item Actualmente utilizamos una arquitectura monolítica para el desarrollo, la cual no permite mucha escalabilidad, así que podríamos migrarlo todo a microservicios y a la nube para que sea más escalable.
\end{itemize}
En cuanto a la posible extensión en el ámbito de las matemáticas podemos destacar:
\begin{itemize}
	\item Diseñar algoritmos numéricos para la resolución de S.E.I.V (sistemas de ecuaciones integrales de Volterra), lineales o no.
	\item Utilizar otro modelo que involucre S.E.I.V, utilizando un desarrollo análogo resolviéndolo con un software si se trata de un sistema lineal, o con un algoritmo numérico si es no lineal, ya que los software son limitados a la hora de manejar sistemas no lineales.
\end{itemize}
\endinput
%------------------------------------------------------------------------------------
% FIN DEL CAPÍTULO. 
%------------------------------------------------------------------------------------