% !TeX root = ../libro.tex
% !TeX encoding = utf8
\chapter{Sistemas de ecuaciones diferenciales lineales}
MEJORAR ESCRITURA PERO PRIMERO LO PONGO A PELO.
Los sistemas de ecuaciones diferenciales lineales son un caso particular de los sistemas de ecuaciones de Volterra lineales. Vamos a ver esto de forma más clara tanto escalar como vectorialmente. XD
\section{Escalar}
Consideramos el PVI:
\begin{equation}
	x \in \mathcal{C}^1(0,B):\left\lbrace\begin{array}{c} x'(t) = a(t)x(t)+b(t) \\ x(0) = x_0 \end{array}\right.,\qquad t \in (0,B)
\end{equation}
La solución del PVI \begin{equation}
	x \in \mathcal{C}[0,B]: x = T(x)
\end{equation}
donde definimos el operador $T$ como
\begin{equation}
	T: \mathcal{C}[0,B] \longrightarrow \mathcal{C}[0,B]
\end{equation}
\begin{equation}
	\qquad \qquad x \longmapsto T(x): [0,B] \rightarrow \R
\end{equation}
\begin{equation}
	\qquad \qquad \qquad \qquad \qquad \qquad \qquad \qquad (T(x))(t) = x_0 + \int_0^t(a(s)x(s)+b(s))ds
\end{equation}

En definitiva, la solución del PVI coincide con la solución de la ecuación integral lineal:
\begin{equation}
	x(t) = x_0 + \int_0^t (a(s)x(s)+b(s))ds
\end{equation}
y vemos que es un caso particular de la ecuación de Volterra de segunda clase, ya que
\begin{equation}
	x(t) = f(t) + \lambda \int_0^t K(t,s)x(s)ds,
\end{equation}
donde
\begin{equation}
	f(t) = x_0 + \int_0^t b(s)ds, \qquad K(t,s) = a(s), \qquad \lambda = 1.
\end{equation}
¿Ver que es una equivalencia? es decir, para el otro lado
\section{Vectorial}

\endinput
%------------------------------------------------------------------------------------
% FIN DEL CAPÍTULO. 
%----------------------------------------------------------------------------------
-