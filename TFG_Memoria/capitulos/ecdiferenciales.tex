% !TeX root = ../libro.tex
% !TeX encoding = utf8
\chapter{Equivalencia entre ecuaciones diferenciales y ecuaciones de Volterra}
A continuación veremos la equivalencia que existe entre ambos tipos de ecuaciones, más concretamente, veremos las ecuaciones diferenciales lineales como un caso particular de las ecuaciones de Volterra. Primero vamos a ver el caso escalar, y después lo haremos vectorialmente.
\section{Escalar}
Consideramos el PVI:
\begin{equation}
	x \in \mathcal{C}^1(0,B):\left\lbrace\begin{array}{c} x'(t) = a(t)x(t)+b(t) \\ x(0) = x_0 \end{array}\right.,\qquad t \in (0,B)
\end{equation}
La solución del PVI es de la forma
\begin{equation}
	x \in \mathcal{C}[0,B]: x = T(x)
\end{equation}
donde definimos el operador lineal $T$ como
\begin{equation}
	T: \mathcal{C}[0,B] \longrightarrow \mathcal{C}[0,B]
\end{equation}
\begin{equation}
	\qquad \qquad x \longmapsto T(x): [0,B] \rightarrow \R
\end{equation}
\begin{equation}
	\qquad \qquad \qquad \qquad \qquad \qquad \qquad \qquad (T(x))(t) = x_0 + \int_0^t(a(s)x(s)+b(s))ds
\end{equation}
Por tanto, vemos que la solución del PVI es un caso particular de la ecuación integral lineal de Volterra de segunda clase:
\begin{equation}
	x(t) = f(t) + \lambda \int_0^t K(t,s)x(s)ds,
\end{equation}
donde
\begin{equation}
	f(t) = x_0 + \int_0^t b(s)ds, \qquad K(t,s) = a(s), \qquad \lambda = 1.
\end{equation}
Para ver la equivalencia, aplicamos el Teorema Fundamental del Cálculo a la siguiente ecuación integral:
\begin{equation}
	x(t) = x_0 + \int_0^t (a(s)x(s)+b(s))ds,
\end{equation}
y obtenemos
\begin{equation}
	\left\lbrace\begin{array}{c} x'(t) = a(t)x(t)+b(t) \\ x(0) = x_0 \end{array}\right.,\qquad t \in (0,B).
\end{equation}
En definitiva, la solución del PVI inicial coincide con la solución de la ecuación de Volterra lineal de segunda clase.
\section{Vectorial}
En este caso, el PVI sería:
\begin{equation}\label{eq:vect}
 \hspace*{-3cm} 	\begin{array}{c} x \in \mathcal{C}^1([0,B],\R^n) \\ x:[0,B] \longrightarrow \R^n \\ \qquad \qquad \qquad \qquad t \longmapsto x(t) = (x_1(t),...,x_n(t)) \end{array}:\left\lbrace\begin{array}{c} x_1'(t) = a_{11}(t)x_1(t)+...+a_{1n}(t)x_n(t)+b_1(t) \\ ... \\ x_n'(t) = a_{n1}(t)x_1(t)+...+a_{nn}(t)x_n(t)+b_n(t) \\ x_1(0) = x_1,...,x_n(0) =x_n, \end{array}\right.
\end{equation}
utilizando notación matricial para simplificar las fórmulas, nos quedaría de la siguiente forma:
\begin{equation}
	\begin{array}{c}
		\textbf{x}'(t) = \textbf{A}(t)\textbf{x}(t)+\textbf{b}(t) \\ \textbf{x}(0) = \textbf{x}_\textbf{0}
	\end{array}
\end{equation}
donde
\begin{equation}
	\textbf{A}(t) = \begin{pmatrix}
		a_{11}(t) & \cdots & a_{1n}(t)\\ 
		\vdots & & \vdots \\
		a_{n1}(t) & \cdots & a_{nn}(t)
	\end{pmatrix}, \qquad \textbf{x}(t) = \begin{pmatrix}
	x_1(t) \\ \vdots \\ x_n(t)
	\end{pmatrix}, \qquad \textbf{b}(t) = \begin{pmatrix}
	b_1(t) \\ \vdots \\ b_n(t)
	\end{pmatrix}
\end{equation}
y el vector $\textbf{x}_\textbf{0}$ está formado por los valores iniciales de cada una de las ecuaciones.

La solución de este PVI viene dada por
\begin{equation}
	\textbf{x} \in \mathcal{C}([0,B],\R^n): \qquad \textbf{x} = \textbf{T(x)}
\end{equation}
donde 
\begin{equation}
	\begin{array}{c}
		\textbf{T}: (\mathcal{C}[0,B], \R^n) \longrightarrow \mathcal{C}([0,B],\R^n) \\  \textbf{x} \longmapsto \textbf{Tx}:[0,B] \longrightarrow \R^n \\  \qquad \qquad \qquad \qquad \qquad \textbf{Tx}(t)= \textbf{x}_\textbf{0} + \int_0^t (\textbf{A}(s)\textbf{x}(s)+\textbf{b}(s))ds,
	\end{array}
\end{equation}
cuando hablamos de hacer una integral compuesta por vectores, la haremos componente a componente, es decir,
\begin{equation}
	\int_0^t (\textbf{A}(s)\textbf{x}(s)+\textbf{b}(s))ds = \int_0^t \textbf{y}(s)ds = \begin{pmatrix}
	\displaystyle	\int_0^t y_1(s)ds \\ \displaystyle \int_0^t y_2(s)ds \\ \vdots \\ \displaystyle \int_0^t y_n(s)ds,
	\end{pmatrix}
\end{equation}
donde $\textbf{y}(s) = (y_1(s),y_2(s),...,y_n(s))$. Luego al igual que en el caso escalar, la solución del PVI es un caso particular de la ecuación de Volterra de segunda clase
\begin{equation}
	\textbf{x}(t) = \textbf{f}(t) + \int_0^t \textbf{K}(t,s)\textbf{x}(s)ds.
\end{equation}
Ahora vemos la equivalencia aplicando el Teorema Fundamental del Cálculo a cada uno de los elementos de la matriz y de los vectores de la siguiente ecuación:
\begin{equation}
	\textbf{x}(t)= \textbf{x}_\textbf{0} + \int_0^t (\textbf{A}(s)\textbf{x}(s)+\textbf{b}(s))ds,
\end{equation}
obteniendo:
\begin{equation}
	\begin{array}{c}
		\textbf{x}'(t) = \textbf{A}(t)\textbf{x}(t)+\textbf{b}(t) \\ \textbf{x}(0) = \textbf{x}_\textbf{0}.
	\end{array}
\end{equation}
Es decir, tanto de forma escalar como vectorial, tenemos el siguiente resultado.
\begin{corolario}
	La solución del PVI~\eqref{eq:vect} coincide con la solución de la ecuación integral lineal de Volterra de segunda clase dada por
	\begin{equation}
		\textbf{x}(t)= \textbf{x}_\textbf{0} + \int_0^t (\textbf{A}(s)\textbf{x}(s)+\textbf{b}(s))ds.
	\end{equation}
\end{corolario}

Vamos a ver un ejemplo en el cual, partiendo de un PVI formado por un sistema de ecuaciones, lo transformaremos en un sistema de ecuaciones integrales de Volterra, y posteriormente, lo resolveremos mediante el método de aproximaciones sucesivas, luego lo volveremos a resolver utilizando el programa \textit{maxima}, y por último, compararemos los resultados obtenidos.
\begin{ejemplo}
	Consideramos el PVI
	\begin{equation}
		\left\lbrace\begin{array}{c} x'(t) = -2x(t)-4y(t) + 4t +1 \\ y'(t) = -x(t) + y(t) +\dfrac{3}{2}t^2 \\ x(0) = 0, \qquad y(0) = 1 \end{array}\right., \qquad t \in [0,1],
	\end{equation}
	que, como hemos visto anteriormente, es equivalente a la siguiente ecuación de Volterra:
	\begin{equation}
		\textbf{x}(t)= \textbf{x}_\textbf{0} + \int_0^t (\textbf{A}(s)\textbf{x}(s)+\textbf{b}(s))ds.
	\end{equation}
	donde
	\begin{equation}
		\textbf{A}(t) = \begin{pmatrix}[1.5]
			-2 & -4 \\ 
			-1 & 1
		\end{pmatrix}, \qquad \textbf{x}(t) = \begin{pmatrix}
			x(t) \\ y(t)
		\end{pmatrix}, \qquad \textbf{b}(t) = \begin{pmatrix}[1.7]
			4t+1 \\ \dfrac{3}{2}t^2
		\end{pmatrix}, \qquad \textbf{x}_\textbf{0} = \begin{pmatrix}
		0 \\ 1
		\end{pmatrix},
	\end{equation}
	así, aplicaremos la siguiente fórmula iterativa para calcular las aproximaciones sucesivas:
	\begin{equation}
		\textbf{x}_{n+1}(t) = \textbf{f}(t) + \int_0^t \textbf{K}(s,t)\textbf{x}_n(s)ds, \qquad n \geqslant 0.
	\end{equation}
	donde
	\begin{equation}
		\textbf{f}(t) = \textbf{x}_\textbf{0} + \int_0^t \textbf{b}(s)ds, \qquad \textbf{K}(t,s) = \textbf{A}(s),
	\end{equation}
	sustituimos $\textbf{x}_\textbf{0}$ y obtenemos:
	\begin{equation}
		\textbf{x}_{1}(t) = \textbf{f}(t) + \int_0^t \textbf{K}(s,t)\textbf{x}_0(s)ds = \begin{pmatrix}[1.7]	2t^2 - 3t \\ \dfrac{t^3}{2} +t+1	\end{pmatrix},
	\end{equation}
	en la segunda iteración tenemos
	\begin{equation}
		\textbf{x}_{2}(t) = \textbf{f}(t) + \int_0^t \textbf{K}(s,t)\textbf{x}_1(s)ds = \begin{pmatrix}[1.7] -2t^4-4t^3+4t^2-3t\\ \dfrac{t^4}{2}-\dfrac{3t^3}{2}+4t^2+t+1	\end{pmatrix},
	\end{equation}
	Así, llegamos a que la iteración número cinco es:
	\begin{equation}
		\textbf{x}_5(t) = \begin{pmatrix}[1.7]
			26t^7 + 122t^6 - 190t^5 + 46t^4 - 24t^3 + 4t^2-3t   \\ \dfrac{29t^7}{2}+\dfrac{53t^6}{2}-\dfrac{43t^5}{2}+\dfrac{49t^4}{2}+\dfrac{t^3}{2} +4t^2+t+1
		\end{pmatrix}
	\end{equation}
\end{ejemplo}
\endinput
%------------------------------------------------------------------------------------
% FIN DEL CAPÍTULO. 
%----------------------------------------------------------------------------------
-