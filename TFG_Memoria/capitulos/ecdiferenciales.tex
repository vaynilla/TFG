% !TeX root = ../libro.tex
% !TeX encoding = utf8
\chapter{Equivalencia entre ecuaciones diferenciales y ecuaciones de Volterra}
A continuación veremos la equivalencia que existe entre ambos tipos de ecuaciones, más concretamente, veremos las ecuaciones diferenciales lineales como un caso particular de las ecuaciones de Volterra. Primero vamos a ver el caso escalar, y después lo haremos vectorialmente.
\section{Escalar}
Consideramos el PVI:
\begin{equation}
	x \in \mathcal{C}^1(0,B):\left\lbrace\begin{array}{c} x'(t) = a(t)x(t)+b(t) \\ x(0) = x_0 \end{array}\right.,\qquad t \in (0,B)
\end{equation}
La solución del PVI es de la forma
\begin{equation}
	x \in \mathcal{C}[0,B]: x = T(x)
\end{equation}
donde definimos el operador lineal $T$ como
\begin{equation}
	T: \mathcal{C}[0,B] \longrightarrow \mathcal{C}[0,B]
\end{equation}
\begin{equation}
	\qquad \qquad x \longmapsto T(x): [0,B] \rightarrow \R
\end{equation}
\begin{equation}
	\qquad \qquad \qquad \qquad \qquad \qquad \qquad \qquad (T(x))(t) = x_0 + \int_0^t(a(s)x(s)+b(s))ds
\end{equation}
Por tanto, vemos que la solución del PVI es un caso particular de la ecuación integral lineal de Volterra de segunda clase:
\begin{equation}
	x(t) = f(t) + \lambda \int_0^t K(t,s)x(s)ds,
\end{equation}
donde
\begin{equation}
	f(t) = x_0 + \int_0^t b(s)ds, \qquad K(t,s) = a(s), \qquad \lambda = 1.
\end{equation}
Para ver la equivalencia, aplicamos el Teorema Fundamental del Cálculo a la siguiente ecuación integral:
\begin{equation}
	x(t) = x_0 + \int_0^t (a(s)x(s)+b(s))ds,
\end{equation}
y obtenemos
\begin{equation}
	\left\lbrace\begin{array}{c} x'(t) = a(t)x(t)+b(t) \\ x(0) = x_0 \end{array}\right.,\qquad t \in (0,B).
\end{equation}
En definitiva, la solución del PVI inicial coincide con la solución de la ecuación de Volterra lineal de segunda clase.
\section{Vectorial}
INDICACIÓN:
En este caso, el PVI sería (y lo pongo entero), luego lo escribo de forma matricial y vectorial con las negritas. Después hago la misma equivalencia que en el caso escalar. Luego la conclusión entiendo que es la misma de forma vectorial? DUDA
\endinput
%------------------------------------------------------------------------------------
% FIN DEL CAPÍTULO. 
%----------------------------------------------------------------------------------
-