% !TeX root = ../libro.tex
% !TeX encoding = utf8

\setchapterpreamble[c][0.75\linewidth]{%
	\sffamily
  Podemos encontrar ecuaciones integrales de muchos tipos distintos. Esto depende principalmente de los límites de integración y del núcleo de la ecuación. A continuación nos enfocaremos en los tipos de ecuaciones integrales.
	\par\bigskip
}
\chapter{Clasificación de ecuaciones integrales}\label{ch:primer-capitulo}

\section{Introducción}
El objetivo que perseguimos en este primer capitulo es el estudio y clasificación de ecuaciones integrales, para proporcionar una comprensión más profunda de su estructura y comportamiento, y aunque no siempre se indique de forma explícita, \textbf{todas las ecuaciones integrales con las que vamos a trabajar son lineales}. A través de un análisis detallado y riguroso, vamos a establecer un marco teórico sólido que permita abordar con mayor eficacia una amplia gama de problemas en contextos prácticos. Además, ilustraremos ejemplos que nos ayudarán a entender más fácilmente la diferencia entre los distintos tipos de ecuaciones integrales.

El modelo de ecuación integral lineal con el que vamos a trabajar es el siguiente:
\begin{equation}\label{eq:ec1}
	u(x) = f(x) + \int_{g(x)}^{h(x)} K(x,t)u(t)dt,
\end{equation}
donde $g(x)$ y $h(x)$ son los límites de integración, y $K(x,t)$ es una función conocida de dos variables reales, \textit{x} y \textit{t}, que llamaremos el \textit{núcleo} de la ecuación integral. Podemos ver cómo la función real desconocida $u(x)$, que queremos determinar,  aparece tanto dentro como fuera de la integral, sin embargo, podría aparecer sólo dentro de la misma, siendo así, una ecuación de primera clase. En este trabajo sólo nos centraremos en las ecuaciones de segunda clase como la que tenemos en~\eqref{eq:ec1}.

Las funciones $f(x)$ y $K(x,t)$ son conocidas de antemano, y los límites de integración pueden ser ambos variables, constantes, o uno variable y otro constante. Esto será clave para determinar el tipo de ecuación integral que tenemos delante.\\
\begin{observacion}
	Nótese que si la función $f(x)$ es idénticamente nula, diremos que la ecuación resultante 
	\begin{equation}\label{}
		u(x) = \int_{g(x)}^{h(x)} K(x,t)u(t)dt
	\end{equation}
	es \textit{homogénea}.
\end{observacion}
Señalemos finalmente que hemos tomado \cite{WazWaz} como referencia para esta primera sección.
\section{Ecuaciones integrales de Volterra}
En las ecuaciones integrales de Volterra, uno de los límites de integración es una variable y el otro una constante. Para las ecuaciones integrales de Volterra de \textit{segunda clase}, la función desconocida $u(x)$ aparece tanto dentro como fuera de la integral. Se representa de la siguiente forma:
\begin{equation}\label{eq:volterra}
	u(x) = f(x) + \int_0^x K(x,t)u(t)dt, \qquad x \in [0,B].
\end{equation}
\begin{ejemplo}
	A continuación mostramos un ejemplo de una ecuación integral de Volterra de segunda clase:
	\begin{equation}\label{}
		u(x) = 5x + 2\int_0^x (x-t)u(t)dt, \qquad x \in [0,4].
	\end{equation}	
\end{ejemplo}
<<<<<<< Updated upstream
\subsection{Ecuaciones integrales singulares}
=======
Para continuar vamos a introducir el concepto de ecuación integral de Volterra singular:
>>>>>>> Stashed changes
\begin{definicion}
	Diremos que una ecuación integral de Volterra
	\begin{equation}
		u(x) = f(x) + \int_0^x K(x,t)u(t)dt, \qquad x \in [0,B]
	\end{equation}
	es \textit{singular} si satisface alguna de las siguientes condiciones:
	\begin{itemize}
		\item Uno de los límites de integración, o ambos, son infinitos.
		\item El núcleo $K(x,t)$ no está acotado en algún punto del intervalo de integración.
	\end{itemize}
\end{definicion}
Vamos a centrarnos en una ecuación particular de segunda clase:
\begin{equation}\label{}
	u(x) = f(x) + \int_0^x \dfrac{1}{(x-t)^\alpha}u(t)dt, \qquad 0 < \alpha < 1, \qquad x \in [0,B].
\end{equation}
Se conoce como \textit{ecuación integral singular débil}. Nótese que el núcleo no está acotado en el límite superior $t = x$. 
\begin{ejemplo}
	Una ecuación integral singular débil sería:
	\begin{equation}\label{}
		\dfrac{\sin x}{2} = \int_0^x \dfrac{1}{\sqrt{x-t}}u(t)dt,
	\end{equation}
\end{ejemplo}

\section{Ecuaciones integrales de Fredholm}
En las ecuaciones integrales de Fredholm, los límites de integración son fijos. Además, al igual que con las ecuaciones integrales de Volterra, al estudiar las ecuaciones de segunda clase, la función desconocida $u(x)$ aparece tanto dentro como fuera de la integral. De igual forma que hemos hecho con las ecuaciones de Volterra, vamos a definir la ecuación integral de Fredholm de \textit{segunda clase}:
\begin{equation}\label{}
	u(x) = f(x) + \int_a^b K(x,t)u(t)dt, \qquad x \in [a,b].
\end{equation}
\begin{ejemplo}
	Un ejemplo de una ecuación integral de Fredholm de segunda clase puede ser el siguiente:
	\begin{equation}\label{}
		u(x) = \sin x + \dfrac{1}{4}\int_{0}^2 (x-t)u(t)dt, \qquad x \in [0,2].
	\end{equation}
\end{ejemplo}

\section{Ecuaciones integrales de Fredholm}
En las ecuaciones integrales de Fredholm, los límites de integración son fijos. Además, al igual que con las ecuaciones integrales de Volterra, al estudiar las ecuaciones de segunda clase, la función desconocida $u(x)$ aparece tanto dentro como fuera de la integral. De igual forma que hemos hecho con las ecuaciones de Volterra, vamos a definir la ecuación integral de Fredholm de \textit{segunda clase}:
\begin{equation}\label{}
	u(x) = f(x) + \int_a^b K(x,t)u(t)dt, \qquad x \in [a,b].
\end{equation}
\begin{ejemplo}
	Un ejemplo de una ecuación integral de Fredholm de segunda clase puede ser el siguiente:
	\begin{equation}\label{}
		u(x) = \sin x + \dfrac{1}{4}\int_{0}^2 (x-t)u(t)dt, \qquad x \in [0,2].
	\end{equation}
\end{ejemplo}

\section{Ecuaciones integrales de Volterra-Fredholm}
Como curiosidad, estas ecuaciones surgieron del modelo matemático del desarrollo espacio-tiempo de una epidemia, y de varios modelos físicos y biológicos (véase \cite{WazWaz}). Normalmente nos aparecen representadas así:
\begin{equation}\label{eq:volt-fred1}
	u(x) = f(x) + \int_a^x K_1(x,t)u(t)dt + \int_a^b K_2(x,t)u(t)dt, \qquad x \in [a,b].
\end{equation}
Es interesante ver que~\eqref{eq:volt-fred1} contiene ecuaciones integrales disjuntas de Volterra y Fredholm, ya que la primera integral tiene un límite variable y la segunda tiene ambos límites fijos. Además, podemos ver que la función $u(x)$ aparece dentro y fuera de la integral, como ocurre en las ecuaciones de segunda clase.
\begin{ejemplo}
	Un ejemplo de una ecuación integral de Volterra-Fredholm es el siguiente:
	\begin{equation}\label{}
		u(x) = 9x^3 - 2x + \int_0^x xu(t)dt - \int_{0}^1 tu(t)dt, \qquad x \in [0,1].
	\end{equation}
\end{ejemplo}

\endinput