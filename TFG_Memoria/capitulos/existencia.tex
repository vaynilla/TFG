% !TeX root = ../libro.tex
% !TeX encoding = utf8
\chapter{Existencia y unicidad de solución de la ecuación lineal integral de Volterra de segunda clase}
\section{Introducción}
En esta sección nos hemos basado en el texto de \cite{Atkinson}, aunque como se verá, hemos realizado algunas modificaciones como en la demostración del teorema de la serie geométrica.

En la sección anterior hemos presentado una amplia gama de ecuaciones integrales, pero ahora estudiamos con detalle las lineales de Volterra de segunda clase, ya que serán las que permitan describir y resolver satisfactoriamente el modelo de distribución de la temperatura interna de un edificio.\\ Si reparamos en la propia naturaleza de este tipo de ecuaciones, debemos exigir cierto grado de regularidad (aunque sólo sea continuidad, incluso algún tipo de buen comportamiento integral) a las funciones dato y a la solución. Todo ello se detalla en este epígrafe.
\section{Teorema de la serie geométrica y sus variantes}
Podemos encontrar la aplicación de este teorema habitualmente en análisis numérico y análisis funcional además de matemática aplicada, esto se debe a que es una gran herramienta para analizar si tienen solución los problemas cercanos a otros problemas para los que sí podemos asegurar la existencia de una solución única.

Primero vamos a introducir un resultado muy importante que utilizaremos posteriormente para demostrar este teorema, que garantiza la existencia y unicidad de puntos fijos de ciertas funciones definidas sobre espacios métricos y proporciona un método para encontrarlos.
\begin{definicion}
	Sea $X$ un espacio de Banach y $T:X \rightarrow X$ una aplicación. Se dice que $T$ es \textit{contractiva} si existe una constante $c$ verificando $0 \leqslant c < 1$ tal que $\lVert T(x) - T(y) \rVert \leqslant c \lVert x-y \rVert$, para cualesquiera $x, y \in X$.
\end{definicion}
\begin{teorema}
	(Teorema del punto fijo de Banach) Sea $X$ un espacio de Banach y sea $T:X \rightarrow X$ una aplicación contractiva en $X$. Entonces existe un único punto fijo de $T$.\\
	Además, para todo punto $x_0$ de $X$, la sucesión $\{T^n(x_{0})\}_{n=0}^\infty$ converge a dicho punto fijo.
\end{teorema}
Este teorema tiene sentido incluso para espacios métricos completos, pero nosotros no necesitamos ese nivel de generalización. 

Como notación, utilizaremos $\mathcal{L}(X)$ para referirnos al espacio de todos los operadores lineales de $X$ en sí mismo, de igual forma $\mathcal{L}(X,Y)$ será el espacio de todos los operadores lineales de $X$ en $Y$, ambos dotados de su norma usual. Ahora sí, podemos enunciar y demostrar el teorema de la serie geométrica:
\begin{teorema}\label{teorema}
	Sea $X$ un espacio de Banach y $L \in \mathcal{L}(X)$. Suponemos
	\begin{equation}
		\lVert L \rVert < 1.
	\end{equation}
	Entonces $I - L$ es una biyección en $X$, su inversa es un operador lineal y continuo,
	\begin{equation}
		(I-L)^{-1} = \sum_{n=0}^{\infty}L^n,
	\end{equation}
	y
	\begin{equation}\label{eq:teo1}
		\lVert (I-L)^{-1} \rVert \leqslant \dfrac{1}{1 - \lVert L \rVert}.
	\end{equation}
\end{teorema}
\begin{proof}
	Sea $y \in X$, definimos el operador 
	\begin{equation}
		T: X \rightarrow X, \qquad T(x) = y+Lx.
	\end{equation}
	Vamos a ver que $T$ es contractivo: sean $u, v \in X$, y sea $c = \lVert L \rVert$, $0 \leqslant c < 1$; entonces 
	\begin{equation}
		\lVert Tu - Tv \rVert = \lVert y + Lu - (y + Lv) \rVert = \lVert Lu - Lv \rVert = \lVert L(u-v) \rVert \leqslant c\lVert u-v \rVert.
	\end{equation}
	Nótese que en la tercera igualdad hemos aplicado la linealidad de $L$ y en la desigualdad su continuidad. Ahora aplicamos el teorema del punto fijo de Banach y tomamos como punto inicial $x_0 = y$, por tanto tenemos que
	\begin{equation}
		x_n = y + Lx_{n-1} = \sum_{j=0}^{n}L^jy, \qquad n \geqslant 1
	\end{equation}
	converge al único punto fijo de $T$, $x = Tx$. Es decir,
	\begin{equation}
		x = y + Lx \Rightarrow (I-L)x=y,
	\end{equation}
	y además $x=\displaystyle\sum_{j=0}^{\infty}L^jy$, como queríamos demostrar. Ahora vamos a ver que $(I-L)$ es biyectivo y continuo:
	\begin{itemize}
		\item Ya hemos visto que es sobreyectiva, pues $y \in X$ es arbitrario.
		\item Inyectiva: 
		\begin{equation}
			\begin{split}
				(I-L)u = (I-L)v \Rightarrow 0 = \lVert u-Lu-v+Lv & \rVert \geqslant \lVert u-v \rVert - \lVert L(u-v) \rVert \geqslant \\
				  \geqslant \lVert u-v \rVert - \lVert L \rVert \lVert u-v \rVert = (1 - \lVert L \rVert) \lVert u-v  \rVert &\Rightarrow u = v.
			\end{split}
		\end{equation}
		(Puesto que $(1 - \lVert L \rVert) > 0$).
		\item El teorema de isomorfismos de Banach (véase \cite{brezis}) nos asegura la continuidad.
	\end{itemize}
	A la vista de lo anterior, 
	\begin{equation}
		(I-L)^{-1}=\sum_{j=0}^{\infty}L^j.
	\end{equation}
	Por último, para probar~\eqref{eq:teo1}, teniendo en cuenta una vez más que $\lVert L \rVert < 1$, tenemos que
	\begin{equation}
		\lVert (I-L)^{-1} \rVert \leqslant \sum_{j=0}^{\infty}\lVert L^j\rVert \leqslant \sum_{j=0}^{\infty}\lVert L\rVert^j = \dfrac{1}{1-\lVert L \rVert}.
	\end{equation}
\end{proof}
Hay una demostración alternativa utilizando la complitud del espacio y la suma de una serie geométrica (véase \cite{Atkinson}).
\begin{observacion}
	El teorema dice que bajo las hipótesis que hemos establecido, para cualquier $f \in X$, la ecuación
	\begin{equation}\label{eq:teo2}
		(I-L)u = f
	\end{equation}
	tiene solución única $u = (I-L)^{-1}f \in X$. Además, la solución depende continuamente de la parte derecha $f$: siendo $(I-L)u_1 = f_1$ y $(I-L)u_2 = f_2$, de esto se sigue que
	\begin{equation}
		u_1 - u_2 = (I-L)^{-1}(f_1 - f_2),
	\end{equation}
	y por tanto,
	\begin{equation}
		\lVert u_1 - u_2 \rVert \leqslant c \lVert f_1 - f_2 \rVert
	\end{equation}
	con $c = 1/(1 - \lVert L \rVert)$.
\end{observacion}
\begin{observacion}
	Este teorema también nos da una forma de aproximar la solución de la ecuación~\eqref{eq:teo2}. Bajo las hipótesis del teorema, tenemos
	\begin{equation}
		u = \lim_{n \rightarrow \infty}u_n
	\end{equation}
	donde
	\begin{equation}
		u_n = \sum_{j=0}^{n}L^jf.
	\end{equation}
\end{observacion}
Si aproximamos la solución $u$ por una suma parcial $u_n$, el error cometido puede estimarse explícitamente:
\begin{equation}
	\lVert u-u_n \rVert = \lVert \sum_{j=n+1}^\infty L^jf \rVert \leqslant \sum_{j=n+1}^\infty \lVert L \rVert^j \lVert f \rVert = \dfrac{\lVert L \rVert^{n+1}}{1-\lVert L \rVert}\lVert f \rVert.
\end{equation}
Como consecuencia del \autoref{teorema} presentamos un resultado importante que nos garantiza la existencia y unicidad de solución para una ecuación lineal integral de Fredholm de segunda clase.
\begin{corolario}\label{ej:1}
	Sea la ecuación integral de Fredholm de segunda clase
	\begin{equation}
	u(x) = f(x)+ \int_{a}^{b}K(x,t)u(t)dt, \qquad x \in [a,b],
	\end{equation}
	donde $K$ y $f$ son continuas y $\lVert K \rVert_{\infty} < \dfrac{1}{b-a}$, entonces la ecuación tiene solución única $u \in \mathcal{C}[a,b]$.
	\begin{proof}
		Sea $X = \mathcal{C}[a,b]$ con la norma $\lVert \cdot \rVert_\infty$. Vamos a reescribir la ecuación de forma simbólica, organizando los términos para poder aplicar posteriormente el teorema de la serie geométrica:
		\begin{equation}
			(I-\mathcal{K})u = f,
		\end{equation}
		donde $\mathcal{K}$ es un operador integral lineal gerenado por el núcleo $K (\cdot , \cdot )$.
		
		Escribimos la ecuación de esta manera puesto que así se puede convertir en la forma que necesitamos para aplicar el teorema de la serie geométrica:
		\begin{equation}
			(I-L)u = f, \qquad L = \mathcal{K}.
		\end{equation}
		Aplicando el teorema de la serie geométrica, afirmamos que si
		\begin{equation}
			\lVert L \rVert = \lVert \mathcal{K} \rVert < 1,
		\end{equation}
		entonces $(I-L)^{-1}$ existe y
		\begin{equation}
			\lVert(I-L)^{-1}\rVert \leqslant \dfrac{1}{1 - \lVert L \rVert}.
		\end{equation}
		Equivalentemente, si
		\begin{equation}
			\rho := \max_{a \leqslant x \leqslant b} \int_{a}^{x}|K(x,t)|dt < 1,
		\end{equation}
		entonces $(I - \mathcal{K})^{-1}$ existe y 
		\begin{equation}
			\lVert (I - \mathcal{K})^{-1}\rVert \leqslant \dfrac{1}{1 - \rho }.
		\end{equation}
		Como $\lVert K \rVert_{\infty} < \dfrac{1}{b-a}$, entonces $\rho < 1$, luego para cualquier $f \in \mathcal{C}[a,b]$, la ecuación integral tiene solución única $u \in \mathcal{C}[a,b]$ y 
		\begin{equation}
			\lVert u \rVert_\infty \leqslant \lVert (I - \mathcal{K})^{-1} \rVert \lVert f \rVert_\infty \leqslant \dfrac{\lVert f \rVert_\infty}{1 - \rho }.
		\end{equation}
		Nótese que la condición $\rho < 1$ está garantizada en cuanto $\lVert K \rVert_\infty < 1$, siendo $K \in \mathcal{C}([a,b]\times [a,b])$.
	\end{proof}
\end{corolario}
\section{Generalización del teorema}
En principio, a la hora de probar que la ecuación integral de Volterra de segunda clase
\begin{equation}
	u(x) = f(x) + \int_{0}^{x} K(x,t)u(t)dt, \qquad x \in [0,B],
\end{equation}
tiene solución única, y a parte de suponer $B > 0$, que el núcleo $K(x,t) \in \mathcal{C}([0,B]\times[0,B])$ y $f \in \mathcal{C}[0,B]$, si queremos aplicar el teorema de la serie geométrica, tal y como acabamos de comprobar, tendremos que suponer una condición extra sobre el núcleo:
\begin{equation}
	\max_{x\in [0,B]}\int_{0}^{x} |K(x,t)|dt < 1,
\end{equation}
o, de forma más restrictiva, $\lVert K \rVert_{\infty} < \dfrac{1}{B}$. Sin embargo, comprobar esta restricción puede obviarse, gracias a una generalización del teorema de la serie geométrica. Simbólicamente, escribiremos la ecuación integral como $(I-L)u = f$.
\begin{corolario}\label{corolario}
	Sea X un espacio de Banach, $L \in \mathcal{L}(X)$. Suponemos que para algún entero $n \geqslant 1$ se cumple
	\begin{equation}
		\lVert L^n \rVert < 1.
	\end{equation}
	Entonces $I - L$ es una biyección en $X$, su inversa es un operador lineal y continuo,
	\begin{equation}
		(I-L)^{-1}=\sum_{j=0}^{\infty}L^j, 
	\end{equation}
	y
	\begin{equation}\label{eq:teo3}
		\lVert (I-L)^{-1} \rVert \leqslant \dfrac{1}{1 - \lVert L^n \rVert} \sum_{i=0}^{n - 1}\lVert L^i \rVert.
	\end{equation}
\end{corolario}
\begin{proof}
	Gracias al teorema de la serie geométrica, sabemos que $(I-L^n)^{-1}$ existe como un operador biyectivo lineal y continuo de $X$ en sí mismo,
	\begin{equation}
		(I - L^n)^{-1} = \sum_{j=0}^{\infty}L^{nj} 
	\end{equation}
	en $\mathcal{L}(X)$, y
	\begin{equation}
		\lVert (I - L^n)^{-1} \rVert \leqslant \dfrac{1}{1 - \lVert L^n \rVert}.
	\end{equation}
	De las igualdades
	\begin{equation}
		(I - L)(\sum_{i=0}^{n - 1}L^i) = (\sum_{i=0}^{n - 1}L^i)(I-L) = I - L^n,
	\end{equation}
	concluimos que $(I-L)$ es una biyección,
	\begin{equation}
		(I-L)^{-1} = (\sum_{i=0}^{n-1}L^i)(I-L^n)^{-1}
	\end{equation}
	que es continuo, gracias al Teorema de los isomorfismos de Banach, de donde se deduce la desigualdad~\eqref{eq:teo3} y la igualdad
	\begin{equation}
		(I-L)^{-1}=\sum_{j=0}^{\infty}L^j.
	\end{equation}
\end{proof}
Vamos a apoyarnos en esta generalización para ilustrar la demostración del siguiente resultado:
\begin{corolario}\label{coroooolario}
	Toda ecuación integral lineal de Volterra de segunda clase tiene solución única, cuando trabajamos en contexto continuo.
\end{corolario}
\begin{proof}
	Sea la ecuación integral de Volterra de segunda clase:
	\begin{equation}
		u(x) = f(x) + \int_0^x K(x,t)u(t)dt, \qquad x \in [0,B],
	\end{equation}
	donde $f \in \mathcal{C}[0,B]$, $K \in \mathcal{C}([0,B]\times [0,B])$ y $u \in \mathcal{C}[0,B]$ es la función incógnita.
	Definimos el operador lineal $L$, que aparece dentro de la ecuación de Volterra:
	\begin{equation}\label{ref_operador}
		\begin{array}{c}
		L: \mathcal{C}[0,B] \rightarrow \mathcal{C}[0,B] \\
		\qquad \qquad \qquad \qquad \qquad Lu(x) \mapsto \displaystyle \int_{0}^{x} K(x,t)u(t)dt, \qquad x \in [0,B].
		\end{array}
	\end{equation}
	Así, podemos escribir la ecuación como $(I-L)u = f$.
	Vamos a comprobar que verifica las hipótesis de la generalización del teorema de la serie geométrica, es decir, que para algún entero $n \geqslant 1$ se cumple $\lVert L^n \rVert < 1$.\\
	En primer lugar, establecemos la desigualdad
	\begin{equation}
		n \geqslant 1 \Rightarrow \lVert L^n \rVert \leqslant \dfrac{\lVert K \rVert_\infty^nB^n}{n!},
	\end{equation}
	equivalentemente,
	\begin{equation}\label{eq:ref}
		n \geqslant 1, u \in \mathcal{C}[0,B] \Rightarrow \lVert L^nu \rVert_\infty \leqslant \dfrac{\lVert K \rVert_\infty^nB^n}{n!}\lVert u \rVert_\infty.
	\end{equation}
	En efecto, para $n = 1$, si $0 \leqslant x \leqslant B$, entonces
	\begin{equation}
		\begin{split}
			\lvert Lu(x) \rvert & \leqslant \int_{0}^{x} |K(x,t)u(t)|dt \\
			& \leqslant \int_{0}^{x} \lVert K \rVert_\infty \lVert u \rVert_\infty dt\\
			& = \lVert K \rVert_\infty \lVert u \rVert_\infty \int_{0}^{x}dt\\
			& = \lVert K \rVert_\infty \lVert u \rVert_\infty x.
		\end{split}
	\end{equation}
	Así pues,
	\begin{equation}\label{eq:demvolt1}
		|Lu(x)| \leqslant \lVert K \rVert_\infty \lVert u \rVert_\infty x.
	\end{equation}
	Para $n = 2$, si $0 \leqslant x \leqslant B$, tenemos
	\begin{equation}
		\begin{split}
			|L^2u(x)| & \leqslant \int_{0}^{x}|K(x,t)Lu(t)|tdt \\
			& \leqslant \int_{0}^{x} \lVert K \rVert_\infty \lVert K \rVert_\infty \lVert u \rVert_\infty tdt\\
			& =  \lVert K \rVert_\infty^2 \lVert u \rVert_\infty \int_{0}^{x}tdt \\
			& = \lVert K \rVert_\infty^2 \lVert u \rVert_\infty \dfrac{x^2}{2},
		\end{split}
	\end{equation}
	donde en la segunda desigualdad hemos utilizado~\eqref{eq:demvolt1}. Inductivamente deducimos
	\begin{equation}
		\left.\begin{split}
			n \geqslant 1, \qquad & u \in \mathcal{C}[0,B]\\
			& x \in [0,B]
		\end{split}\right\rbrace \Rightarrow |L^nu(x)| \leqslant \dfrac{\lVert K \rVert_\infty^nx^n}{n!}\lVert u \rVert_\infty,
	\end{equation}
	de donde se tiene~\eqref{eq:ref} y, por tanto, 
	\begin{equation}\label{eq:demvolt2}
		\lVert L^n \rVert \leqslant \dfrac{\lVert K \rVert_\infty^nB^n}{n!}.
	\end{equation}
	Debido a que la sucesión $\{\dfrac{\lVert K \rVert_\infty^nB^n}{n!}\}$ converge a $0$ (la serie $\displaystyle \sum_{n=0}^{\infty}\dfrac{\lVert K \rVert_\infty^nB^n}{n!}$ es convergente, $\displaystyle \sum_{n=0}^{\infty}\dfrac{\lVert K \rVert_\infty^nB^n}{n!} = e^{\lVert K \rVert_\infty B}$), la desigualdad~\eqref{eq:demvolt2} da $\lVert L^n \rVert \rightarrow 0$ y, por tanto, a partir de un $n$ en adelante, $\lVert L^n \rVert < 1$.
	Luego podemos aplicar la generalización del teorema de la serie geométrica (\autoref{corolario}) y tenemos que existe el operador inverso $(I-L)^{-1}$ en $\mathcal{C}([0,B])$, y su inverso se puede expresar como la suma de una serie geométrica convergente, por tanto,  la solución única $u$ viene dada por:
	\begin{equation}
		u = (I-L)^{-1}f.
	\end{equation}
\end{proof}
Como $u = \displaystyle \sum_{j=0}^{\infty} L^jf$, una aproximación de dicha solución viene dada por una suma parcial $u_n = \displaystyle \sum_{j=0}^{n}L^jf$ y gracias a~\eqref{eq:demvolt2} tenemos
\begin{equation}
	\lVert u-u_n \rVert \leqslant \sum_{j=n+1}^{\infty}\dfrac{\lVert K \rVert_\infty^j B^j}{j!}.
\end{equation}
\section{Un resultado de perturbación}
Vamos a presentar otra aplicación del teorema de la serie geométrica, distinta a lo hecho con las ecuaciones integrales previamente.
La \textit{perturbación} es una estrategia en matemática aplicada que se enfoca en el estudio de una ecuación al relacionarla con otra ecuación $"$afín$"$, de la cual sabemos que existe un resultado que nos da una solución, esto nos ayuda a encontrar soluciones para problemas más complicados utilizando otros más simples. El siguiente teorema representa una de las herramientas de uso más frecuente en este contexto.

\begin{teorema}
	Sean $X$ e $Y$ espacios normados, siendo al menos uno de ellos completo. Suponemos que $L \in \mathcal{L}(X,Y)$ posee un inverso lineal y continuo $L^{-1}: Y \rightarrow X.$ Además, $M \in \mathcal{L}(X,Y)$ satisface
	\begin{equation}\label{eq:teo7}
		\lVert M-L \rVert < \dfrac{1}{\lVert L^{-1} \rVert}.
	\end{equation}
	Entonces $M:X\rightarrow Y$ es una biyección, $M^{-1} \in \mathcal{L}(Y,X)$ y 
	\begin{equation}\label{eq:teo5}
		\lVert M^{-1} \rVert \leqslant \dfrac{\lVert L^{-1} \rVert}{1 - \lVert L^{-1} \rVert \lVert L-M \rVert}.
	\end{equation}
	Además,
	\begin{equation}\label{eq:teo4}
		\lVert L^{-1} - M^{-1} \rVert \leqslant \dfrac{\lVert L^{-1} \rVert^2 \lVert L-M \rVert}{1 - \lVert L^{-1} \rVert \lVert L-M \rVert}.
	\end{equation}
	Para soluciones de las ecuaciones $Lx_1 = y$ y $Mx_2 = y,$ tenemos la acotación
	\begin{equation}\label{eq:teo6}
		\lVert x_1 - x_2 \rVert \leqslant \lVert M^{-1} \rVert \lVert (L-M)x_1 \rVert.
	\end{equation}
\end{teorema}
\begin{proof}
	Si $Y$ es completo, podemos escribir
	\begin{equation}
		M = [I - (L - M)L^{-1}]L;
	\end{equation}
	mientras que si $X$ es completo, escribiremos
	\begin{equation}
		M = L[I - L^{-1}(L - M)].
	\end{equation}
	Vamos a hacer la demostración en el caso de que $Y$ sea completo.
	El operador $(L-M)L^{-1} \in \mathcal{L}(Y)$ satisface
	\begin{equation}
		\lVert (L - M)L^{-1} \rVert \leqslant \lVert L - M \rVert \lVert L^{-1} \rVert < 1.
	\end{equation}
	Luego por el teorema de la serie geométrica, $[I-(L-M)L^{-1}]^{-1}$ existe, es lineal y continuo y además,
	\begin{equation}
		\lVert [I-(L-M)L^{-1}]^{-1} \rVert \leqslant \dfrac{1}{1 - \lVert (L-M)L^{-1}\rVert} \leqslant \dfrac{1}{1 - \lVert L^{-1} \rVert \lVert L-M \rVert }.
	\end{equation}
	Entonces $M^{-1}$ existe y es lineal y continuo, con
	\begin{equation}
		M^{-1} = L^{-1}[I-(L-M)L^{-1}]^{-1}
	\end{equation}
	y
	\begin{equation}
		\lVert M^{-1} \rVert \leqslant \lVert L^{-1} \rVert \lVert [I-(L-M)L^{-1}]^{-1} \rVert \leqslant \dfrac{\lVert L^{-1} \rVert}{1 - \lVert L^{-1} \rVert \lVert L-M \rVert}.
	\end{equation}
	Para probar~\eqref{eq:teo4}, escribimos
	\begin{equation}
		L^{-1} - M^{-1} = M^{-1}(M-L)L^{-1},
	\end{equation}
	tomamos normas y usamos~\eqref{eq:teo5}.
	
	Para~\eqref{eq:teo6},
	\begin{equation}
		x_1 - x_2 = (L^{-1} - M^{-1})y = M^{-1}(M-L)L^{-1}y = M^{-1}(M-L)x_1
	\end{equation}
	y aplicamos normas.
\end{proof}
Podemos resumir el teorema anterior en una frase: \textit{Un operador cercano a otro operador biyectivo, lineal y continuo, también será biyectivo, lineal y continuo.} Según se ilustra en muchos textos, como \cite{Atkinson}, la aplicación de este resultado es útil para una amplia gama de resultados de existencia de soluciones para ecuaciones integrales y diferenciales.

\begin{ejemplo}
	Vamos a examinar si la ecuación integral
	\begin{equation}\label{eq:ej1}
		\lambda u(x) - \int_{0}^{1} \dfrac{\cos(xt)}{2}u(t)dt = f(x), \qquad 0 \leqslant x \leqslant 1
	\end{equation}
	tiene solución, con $\lambda \neq 0$. De la discusión del corolario~\eqref{ej:1}, si
	\begin{equation}\label{ej:2}
	1 > \lVert K \rVert = \int_{0}^{1}\dfrac{\cos(t)}{2}dt = \sin (1)/2 \approx 0.4207,
	\end{equation}
	entonces para cada $f \in \mathcal{C}[0,1]$, la ecuación admite una solución única $u \in \mathcal{C}[0,1]$.
	
	Para obtener más valores de $\lambda$ para los cuales nuestra ecuación tiene solución única, aplicamos el teorema de perturbación. Ya que $\cos(xt) \approx xt$ para valores pequeños de $|xt|$, comparamos la ecuación con
	\begin{equation}\label{eq:ej2}
		\lambda v(x) - \int_{0}^{1} \dfrac{xt}{2}v(t)dt = f(x), \qquad 0 \leqslant x \leqslant 1.
	\end{equation}
	Siguiendo la notación el teorema de perturbación, la ecuación~\eqref{eq:ej1} sería $Mu = f$, y~\eqref{eq:ej2} sería $Lv = f$. El espacio de Banach es $X = \mathcal{C}[0,1]$ con la norma $\lVert \cdot \rVert_\infty$, y $L,M \in \mathcal{L}(X)$.
	
	Podemos resolver la ecuación integral~\eqref{eq:ej2} de forma explícita. Suponiendo $\lambda \neq 0$, tenemos que cada solución $v$ toma la forma
	\begin{equation}
		v(x) = \dfrac{1}{\lambda}[f(x)+cx]
	\end{equation}
	para alguna constante $c$. Sustituyéndolo en la ecuación nos lleva a una fórmula para $c$, y entonces
	\begin{equation}
		v(x) = \dfrac{1}{\lambda}[f(x) + \dfrac{1}{\lambda - 1/3}\int_{0}^{1}\dfrac{xt}{2}f(t)dt], \qquad \lambda \neq 0,\dfrac{1}{3}.
	\end{equation}
	Esta relación define $L^{-1}f$ para toda $f \in \mathcal{C}[0,1]$.
	
	Para usar el teorema de perturbación, necesitamos medir algunas cantidades. Se puede calcular que
	\begin{equation}
		\lVert L^{-1} \rVert \leqslant \dfrac{1}{|\lambda|}(1 + \dfrac{1}{2|\lambda - 1/3|})
	\end{equation}
	y
	\begin{equation}
		\lVert L-M \rVert = \int_{0}^{1} (t - \dfrac{\cos t}{2})dt = \dfrac{1}{2} - \sin(1)/2 \approx 0.0793.
	\end{equation}
	La condición~\eqref{eq:teo7} viene de
	\begin{equation}\label{ej:3}
		\dfrac{1}{|\lambda|}(1+\dfrac{1}{2|\lambda - 1/3|}) < \dfrac{1}{1/2 - \sin(1)/2}.
	\end{equation}
	Si $\lambda$ es un número real, entonces hay tres casos a considerar: $\lambda > 1/3$, $0 < \lambda < 1/3$, y $\lambda < 0$. Para el caso $\lambda < 0$, la desigualdad es cierta si y sólo si $\lambda < \lambda_0 \approx -0.1596$.
	
	Como consecuencia del teorema de perturbación, tenemos que si $\lambda < \lambda_0$, entonces nuestra ecuación integral tiene solución única para cualquier $f \in \mathcal{C}[0,1]$. Esto es una mejora significativa en comparación con el punto de partida que teníamos~\eqref{ej:2}.
\end{ejemplo}
\section{Existencia y unicidad de solución para sistemas de ecuaciones integrales lineales de Volterra de segunda clase}
Avanzamos ahora un poco más en el estudio de las ecuaciones integrales de Volterra, ocupándonos de un sistema de ecuaciones integrales lineales de Volterra de segunda clase, principalmente porque aparece en el modelo temperatura interna de un edificio que veremos más adelante, trabajaremos en un contexto continuo que, vectorialmente, adopta la expresión 
\begin{equation}
	\textbf{u}(x) = \textbf{f}(x) + \int_0^x \textbf{K}(x,t)\textbf{u}(t)dt, \qquad x \in [0,B],
\end{equation}
donde \textbf{u} es el vector formado por todas las funciones continuas que queremos determinar, \textbf{f} es el vector formado por las funciones continuas independientes, y \textbf{K} es el vector formado por los núcleos continuos de cada una de las ecuaciones que forman el sistema:
\begin{equation}
	\textbf{u}(x) = \begin{pmatrix}	u_1(x) \\ u_2(x) \\ \vdots \\ u_n(x)	\end{pmatrix}, \qquad \textbf{f}(x) = \begin{pmatrix}	f_1(x) \\ f_2(x) \\ \vdots \\ f_n(x)	\end{pmatrix}, \qquad \textbf{K}(x,t) = \begin{pmatrix}	K_1(x,t) \\ K_2(x,t) \\ \vdots \\ K_n(x,t)	\end{pmatrix}.
\end{equation}
\begin{corolario}
	Todo sistema de ecuaciones integrales lineales de Volterra de segunda clase en ambiente continuo tiene solución única.
\end{corolario}
\begin{proof}
	Partimos de la ecuación de Volterra:
	\begin{equation}
		\textbf{u}(x) = \textbf{f}(x) + \int_0^x \textbf{K}(x,t)\textbf{u}(t)dt, \qquad x \in [0,B].
	\end{equation}
	Definimos el operador lineal $L$, que aparece dentro de la ecuación de Volterra:
	\begin{equation}
		\begin{array}{c}
			\textbf{L}: \mathcal{C}([0,B],\R^n) \rightarrow \mathcal{C}([0,B],\R^n) \\
			\qquad \qquad \qquad \qquad \qquad \textbf{L}\textbf{u}(x) \mapsto \displaystyle \int_{0}^{x} \textbf{K}(x,t)\textbf{u}(t)dt, \qquad x \in [0,B].
		\end{array}
	\end{equation}
	Así, podemos escribir la ecuación como $(\textbf{I}-\textbf{L})\textbf{u} = \textbf{f}$.
	Vamos a comprobar que verifica las hipótesis de la generalización del teorema de la serie geométrica, es decir, que para algún entero $n \geqslant 1$ se cumple $\lVert \textbf{L}^n \rVert < 1$.\\
	Establecemos la desigualdad
	\begin{equation}
		n \geqslant 1 \Rightarrow \lVert \textbf{L}^n \rVert_\infty \leqslant \dfrac{\lVert \textbf{K} \rVert_\infty^nB^n}{n!},
	\end{equation}
	equivalentemente,
	\begin{equation}\label{eq:ref2}
		n \geqslant 1, \textbf{u} \in \mathcal{C}([0,B],\R^n) \Rightarrow \lVert \textbf{L}^n\textbf{u} \rVert_\infty \leqslant \dfrac{\lVert \textbf{K} \rVert_\infty^nB^n}{n!}\lVert \textbf{u} \rVert_\infty.
	\end{equation}
	En efecto, para $n = 1$, si $0 \leqslant x \leqslant B$, entonces
	\begin{equation}
		\begin{split}
			\lVert \textbf{L}\textbf{u}(x) \rVert_\infty & \leqslant \int_{0}^{x} \lVert \textbf{K}(x,t)\textbf{u}(t)\rVert_\infty dt \\
			& \leqslant \int_{0}^{x} \lVert \textbf{K} \rVert_\infty \lVert \textbf{u} \rVert_\infty dt\\
			& = \lVert \textbf{K} \rVert_\infty \lVert \textbf{u} \rVert_\infty \int_{0}^{x}dt\\
			& = \lVert \textbf{K} \rVert_\infty \lVert \textbf{u} \rVert_\infty x.
		\end{split}
	\end{equation}
	Así pues,
	\begin{equation}\label{eq:demvolt1_2}
		\lVert \textbf{L}\textbf{u}(x)\rVert_\infty \leqslant \lVert \textbf{K} \rVert_\infty \lVert \textbf{u} \rVert_\infty x.
	\end{equation}
	Para $n = 2$, si $0 \leqslant x \leqslant B$, tenemos
	\begin{equation}
		\begin{split}
			\lVert \textbf{L}^2\textbf{u}(x)\rVert_\infty & \leqslant \int_{0}^{x}\lVert \textbf{K}(x,t)\textbf{L}\textbf{u}(t)\rVert_\infty tdt \\
			& \leqslant \int_{0}^{x} \lVert \textbf{K} \rVert_\infty \lVert \textbf{K} \rVert_\infty \lVert \textbf{u} \rVert_\infty tdt\\
			& =  \lVert \textbf{K} \rVert_\infty^2 \lVert \textbf{u} \rVert_\infty \int_{0}^{x}tdt \\
			& = \lVert \textbf{K} \rVert_\infty^2 \lVert \textbf{u} \rVert_\infty \dfrac{x^2}{2},
		\end{split}
	\end{equation}
	donde en la segunda desigualdad hemos utilizado~\eqref{eq:demvolt1_2}. Inductivamente deducimos
	\begin{equation}
		\left.\begin{split}
			n \geqslant 1, \qquad & \textbf{u} \in \mathcal{C}([0,B],\R^n)\\
			& x \in [0,B]
		\end{split}\right\rbrace \Rightarrow \lVert \textbf{L}^n\textbf{u}(x)\rVert_\infty \leqslant \dfrac{\lVert \textbf{K} \rVert_\infty^nx^n}{n!}\lVert \textbf{u} \rVert_\infty,
	\end{equation}
	de donde se tiene~\eqref{eq:ref2} y, por tanto,
	\begin{equation}\label{eq:demvolt2_2}
		\lVert \textbf{L}^n \rVert_\infty \leqslant \dfrac{\lVert \textbf{K} \rVert_\infty^nB^n}{n!}.
	\end{equation}
	Como la sucesión $\{\dfrac{\lVert \textbf{K} \rVert_\infty^nB^n}{n!}\}$ converge a $0$ (la serie $\displaystyle \sum_{n=0}^{\infty}\dfrac{\lVert \textbf{K} \rVert_\infty^nB^n}{n!}$ es convergente, $\displaystyle \sum_{n=0}^{\infty}\dfrac{\lVert \textbf{K} \rVert_\infty^nB^n}{n!} = e^{\lVert \textbf{K} \rVert_\infty B}$), la desigualdad~\eqref{eq:demvolt2_2} da $\lVert \textbf{L}^n \rVert \rightarrow 0$ y, por tanto, a partir de un $n$ en adelante, $\lVert \textbf{L}^n \rVert < 1$.
	Ahora estamos en condición de aplicar la generalización del teorema de la serie geométrica que vimos anteriormente, ya que aunque estemos en caso vectorial, el resultado es análogo y por tanto sabemos que existe el operador inverso $(\textbf{I}-\textbf{L})^{-1}$ en $\mathcal{C}([0,B],\R^n)$, y además su inverso se puede expresar como la suma de una serie geométrica convergente, por tanto, la solución única $\textbf{u}$ viene dada por:
	\begin{equation}
		\textbf{u} = (\textbf{I}-\textbf{L})^{-1}\textbf{f}.
	\end{equation}
\end{proof}
Las consideraciones hechas en caso escalar relativas a la aproximación $\displaystyle \sum_{j=0}^{n}\textbf{L}^j\textbf{f}$ de la solución $\displaystyle \sum_{j=0}^{\infty}\textbf{L}^j\textbf{f}$ son igualmente válidas aquí, gracias a la acotación~\eqref{eq:demvolt2_2}.

\endinput
%------------------------------------------------------------------------------------
% FIN DEL CAPÍTULO. 
%------------------------------------------------------------------------------------