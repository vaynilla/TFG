% !TeX root = ../libro.tex
% !TeX encoding = utf8
%
%*******************************************************
% Summary
%*******************************************************

\selectlanguage{english}
\chapter{Summary}
The principal objective of this project lies in the creation of a visually intuitive and highly functional tool that allows studying the behavior of temperatures throughout the day in a building. This tool will facilitate the modification of the building’s parameters and structure in a comfortable and dynamic manner, enabling users to observe how temperatures vary in response to these changes. To achieve this, a mathematical model will be applied to predict temperature variations. Additionally, this tool aims to be accessible and beneficial for other users. To support this, an appendix containing a guide for its installation and use has been added, allowing it to be utilized for various purposes, regarding academic, professional, or personal contexts.\\

The main objective of this final degree project in the field of computer science is to design, implement, and deploy a system that, using mathematical models, allows users to simulate and predict the temperature inside a building based on certain parameter configurations. It will also integrate a third-party API (Application Programming Interface) to capture real-time meteorological data (such as external temperature and humidity) and provide one-week forecasts. The system to be developed will follow a Service-Oriented Architecture (SOA) model, offering a set of services to be consumed by the user's application. The entire system will be integrated into a RESTful API and a web platform application. The technologies used, Python for backend development and React for the frontend, will ensure the solution's extensibility, scalability, and usability.

All the specified purposes have been met, except for the integration of the real-time external API, which has been left as future work for a potential extension of the project, considering it excessive for a project of this scope.

In the field of mathematics, the objective is to study some of the methods for solving or approximating solutions to certain integral equations, as well as applying some of these methods to the study of a mathematical model for temperature distribution inside a building with multiple rooms. Some of the concepts, algorithms, and widely used results will be compiled, with a focus on fixed-point methods. Finally, a heating and cooling model for a building will be presented, leading to a problem that will be solved using some of the previously studied techniques. Additionally, various simulations will allow us to realistically analyze a building's energy efficiency in terms of the parameters and functions that describe the model.

All initially proposed objectives in the field of mathematics have been achieved.\\

The project is structured into two main parts: the first one, which involves a more theoretical and mathematical study, and the second, which focuses on computer science with a more practical approach, demonstrating its utility. Both parts complement each other exceptionally well, as the simulator to be implemented is based on the mathematical model developed later in this work. The mathematical model provides a solid theoretical foundation, detailing the equations and principles governing the phenomenon we are studying. This approach allows us not only to validate the simulator but also to adjust its parameters accurately to reflect real conditions. Moreover, the simulator will serve as a practical tool enabling us to visualize and analyze the results obtained from the mathematical model, facilitating data understanding and interpretation. Thus, integrating both parts provides a comprehensive and detailed view of the problem, ensuring greater accuracy and robustness in our conclusions.\\
Additionally, in Appendix A, an installation and access manual for the application is shown so that anyone can access the code and use it.\\

The first part consists of six sections, starting with the classification of the main types of integral equations we will cover. In the second section, we will conduct a more theoretical study on the existence and uniqueness of the solution specifically for the linear Volterra integral equation of the second kind, using the Geometric Series Theorem and some of its variants. Subsequently, in the third and fourth sections, various methods for solving these equations including both scalar and vector forms will be studied and some examples will be provided. The fifth section will illustrate the relationship between differential equations and Volterra equations, allowing us to apply this model to the following section to the use case of building heating and cooling, where we will see many examples of this application.

The second part consists of two sections that begin by setting the theoretical framework related to software architecture, showing the characteristics of the main types of architecture found in applications, along with several examples of current applications using these architectures. Lastly, we will have the design, implementation, and deployment section of the simulator, in which we will show and explain the characteristics of the technologies used, together with the steps in designing and developing the simulator, providing some usage examples.

Finally, there will be a section regarding the conclusions and future work, highlighting the most important aspects of the project, along with possible updates that allow its scalability and improvement, enabling continued work on this topic in the future.



% Al finalizar el resumen en inglés, volvemos a seleccionar el idioma español para el documento
\selectlanguage{spanish} 
\endinput
