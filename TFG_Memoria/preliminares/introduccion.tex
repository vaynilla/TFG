% !TeX root = ../libro.tex
% !TeX encoding = utf8
%
%*******************************************************
% Introducción
%*******************************************************

% \manualmark
% \markboth{\textsc{Introducción}}{\textsc{Introducción}} 

\chapter{Introducción}
La motivación principal de este trabajo radica en la creación de una herramienta visualmente intuitiva y altamente funcional que permita estudiar el comportamiento de las temperaturas a lo largo del día en un edificio. Esta herramienta facilitará la modificación de los parámetros y la estructura del edificio de manera cómoda y dinámica, permitiendo a los usuarios observar cómo varían las temperaturas en respuesta a estos cambios. Para ello, se aplicará un modelo matemático que permitirá realizar las predicciones de estas temperaturas. Además, se busca que esta herramienta sea accesible y beneficiosa para otras personas, se ha añadido como apéndice una guía para la instalación y uso de la misma, permitiéndo ser utilizada para otros fines, ya sea en el ámbito académico, profesional o personal.\\

El objetivo principal de este TFG en el ámbito informático es diseñar, implementar y desplegar un sistema que, haciendo uso de los modelos matemáticos, permita al usuario simular y predecir la temperatura en el interior de un edificio de acuerdo a la configuración de ciertos parámetros. A su vez integrará una API (Application Programming Interface) de terceros para la captura en tiempo real de datos meteorológicos (temperatura y humedad exterior) así como datos dados y predicciones a una semana vista. El sistema a desarrollar seguirá un modelo de Arquitectura Orientada a Servicios (SOA), ofreciendo un conjunto de servicios a consumir por la aplicación del usuario. Se integrará todo el sistema resultando en una API RESTful y una aplicación en una plataforma web. Las tecnologías que vamos a utilizar, Python para el desarrollo del backend y React en el caso del frontend nos garantizan la extensibilidad, escalabilidad y usabilidad de la solución.\\
Se han cumplido todos los objetivos especificados anteriormente, exceptuando la integración de la API externa de tiempo real, que se ha dejado como trabajo a futuro para una posible ampliación del trabajo, debido a que se ha considerado que era excesivo para un trabajo de estas características.

El objeto en el ámbito de las matemáticas consiste en realizar un estudio de algunos de los métodos de resolución o aproximación de la solución para ciertas ecuaciones integrales, así como la aplicación de algunos de ellos al estudio de un modelo matemático para la distribución de la temperatura en el interior de un edificio con varias estancias. Se recopilarán algunos de los conceptos, algoritmos y resultados de uso extendido, incidiendo en los de punto fijo. Finalmente, se presentará un modelo de calentamiento y enfriamiento de un edificio, que conduce a un problema que resolveremos con algunas de las técnicas previamente estudiadas. Además, diversas simulaciones permitirán analizar de manera realista la eficiencia energética de un edificio en términos de los parámetros y funciones que describen el modelo.\\
Se han cumplido todos los objetivos que inicialmente se propusieron en el ámbito de las matemáticas.\\

El trabajo está estructurado en dos partes principales, una primera parte en la cual se realiza un estudio matemático y más teórico, y una segunda parte centrada en la informática, la cual pretende dar un enfoque más práctico y mostrar la utilidad del trabajo. Ambas partes se complementan de manera excepcional, ya que el simulador que implementaremos se fundamenta en el modelo matemático que desarrollaremos más adelante en este trabajo. El modelo matemático proporcionará una base teórica sólida, detallando las ecuaciones y principios que rigen el fenómeno que estamos estudiando. Este enfoque nos permitirá no solo validar el simulador, sino también ajustar sus parámetros de manera precisa para reflejar con exactitud las condiciones reales. Por otro lado, el simulador servirá como una herramienta práctica que nos permitirá visualizar y analizar los resultados obtenidos del modelo matemático, facilitando la comprensión y la interpretación de los datos. Así, la integración de ambas partes nos proporcionará una visión completa y detallada del problema, garantizando una mayor precisión y robustez en nuestras conclusiones.\\
Además, en el Apéndice A, se muestra un manual de instalación y acceso a la aplicación para que cualquier persona pueda acceder al código y utilizarlo.\\

La primera parte consta de seis secciones, comenzando por la clasificación de los principales tipos de ecuaciones integrales que vamos a ver, en la segunda sección realizaremos un estudio más teórico sobre la existencia y unicidad de solución concretamente para la ecuación lineal integral de Volterra de segunda clase, viendo para ello el Teorema de la serie geométrica y algunas de sus variantes. Posteriormente, en la tercera y cuarta sección estudiaremos y daremos ejemplos de varios métodos para resolver estas ecuaciones, tanto de forma escalar como vectorial. La quinta sección ilustrará la relación entre ecuaciones diferenciales y ecuaciones de Volterra, que nos permitirá aplicar este modelo en la sexta sección al caso de uso del calentamiento y enfriamiento de edificios, sección en la cual veremos gran cantidad de ejemplos de esta aplicación.

La segunda parte está compuesta de dos secciones que comienzan sentando el marco teórico relacionado con la arquitectura software, mostrando las características de los principales tipos de arquitectura que podemos encontrar en las aplicaciones junto con varios ejemplos de aplicaciones actuales que utilizan estas arquitecturas. Finalmente tendremos el apartado de diseño, implementación y despliegue del simulador, en el cual mostraremos y explicaremos las características de las tecnologías que hemos utilizado, junto con los pasos en el diseño y desarrollo del simulador, dando algunos ejemplos de su utilización.\\

Por último, un apartado de conclusiones y trabajo futuro, en el cual se destacarán los aspectos más importantes del trabajo, junto a posibles actualizaciones que permiten su escalabilidad y mejora, y permiten seguir trabajando en este tema a futuro.



\endinput
